
\paragraph{Hardware Configuration}

For all our \cnine experiments, we ran \cnine in parallel symbolic execution mode in a heterogeneous cluster environment, with worker CPU frequencies between 2.3--2.6 GHz and with 4--6 GB of RAM available per core.

We performed all \chef experiments on a 48-core 2.3 GHz AMD Opteron 6176 machine with 512 GB of RAM, running Ubuntu 12.04. Each \chef invocation ran on 1 CPU core and used up to 8 GB of RAM on average.

\paragraph{Coverage Measurement}

Line or statement coverage remains widely used, even though its meaningfulness as a metric for test quality is disputed. We measure and report line coverage to give a sense of what users can expect from a test suite generated fully automatically by \cnine or a symbolic execution engine based on \chef.  For Python, we rely on the popular \codebit{coverage} package, and for Lua we use the \codebit{luacov} package.

Since our \chef prototype only supports strings and integers as symbolic program inputs, we count only the lines of code that can be reached using such inputs. We report this number as ``coverable LOC'' in the fifth column of Table~\ref{tab:targets}, and use it in our experiments as a baseline for what such a symbolic execution engine could theoretically cover directly.  For example, for the \codebit{simplejson} library, this includes only code that decodes JSON-encoded strings, not code that takes a JSON object and encodes it into a string. Note that, in principle, such code could still be tested and covered by writing a more elaborate symbolic test that sets up a JSON object based on symbolic primitives~\cite{paas-testing}.

\paragraph{Execution Strategies in \cnine}

On each worker, the underlying \klee engine used the best execution strategies from \cite{klee}, namely an interleaving of random-path and coverage-optimized strategies. At each step, the engine alternately selects one of these heuristics to pick the next state to explore. Random-path traverses the execution tree starting from the root and randomly picks the next descendant node, until a candidate state is reached. The coverage-optimized strategy weighs the states according to an estimated distance to an uncovered line of code, and then randomly selects the next state according to these weights.

%%% Local Variables: 
%%% mode: latex
%%% eval: (visual-line-mode)
%%% fill-column: 1000000
%%% TeX-master: "main"
%%% End:
