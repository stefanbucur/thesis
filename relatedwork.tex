There is extensive work in the area of symbolic execution and automated testing.  In this chapter, we review the use of symbolic execution on commodity software for automated test case generation (Section~\ref{sec:relwork:atcg}), with an emphasis on how previous work addresses the environment problem (Section~\ref{sec:relwork:envproblem}).  We specifically review the work on parallelizing symbolic execution (Section~\ref{sec:relwork:parsymbex}), and discuss existing approaches related to symbolic tests (Section~\ref{sec:relwork:symtests}).

%%%%%%%%%%%%%%%%%%%%%%%%%%%%%%%%%%%%%%%%%%%%%%%%%%%%%%%%%%%%%%%%%%%%%%%%%%%%%%%%
%%%%%%%%%%%%%%%%%%%%%%%%%%%%%%%%%%%%%%%%%%%%%%%%%%%%%%%%%%%%%%%%%%%%%%%%%%%%%%%%

\section{Symbolic Execution for Automated Testing of Commodity Software}
\label{sec:relwork:atcg}

%% We review the most important symbolic execution-based tools.  The story is chronological, illustrating the advancements brought by each tool.

%% King~\cite{king:symbolic:2} described symbolic execution as a practical middle ground between testing a program with a set of concrete inputs and using formal correctness proofs.  The paper introduced EFFIGY, an interactive symbolic execution tool where developers manually step through paths and switch states in the execution tree.  However, the tool could only handle small programs written in a custom language supporting basic integer operations, and had limited performance, due to limited solver capabilities.

Symbolic execution was introduced almost four decades ago as a test case generation technique~\cite{king:symbolic:2, boyer:symbolic}.  The first tools worked on domain-specific languages, supported basic numeric operations, and were mostly used for interactive debugging.  The effectiveness of the technique on more complex programs was limited by the lack of efficient and expressive constraint solvers and by slow hardware.

%% It was only in the past decade that symbolic execution has become a feasible approach to test generation for real-world software, with the advent of modern constraint solvers, such as Chaff~\cite{chaff}, MiniSAT~\cite{minisat}, STP~\cite{stp}, Z3~\cite{Z3}, or CVC~\cite{cvc}, and the exponential increase in hardware performance.  Our tools build on this foundation, too---both Cloud9 and Chef resort to the STP solver for all symbolic queries, such as branch feasibility checks.

The exponential increase in hardware performance and the availability of fast off-the-shelf constraint solvers~\cite{chaff,minisat,stp,Z3,cvc} amplified the research in symbolic execution.  Over the past decade, the new generation of symbolic execution engines produced test suites and found bugs on real-world software ranging from small system utilities to large application suites.

In this section, we highlight the most important tools and discuss the techniques that expanded the applicability of symbolic execution on real-world software.

\subsection{System-level Software Targets}

%% \paragraph{DART}

%% DART automatically extracts the environment interface of a C program using static analysis.  DART augment the classic random fuzzing with a dynamic analysis that collects. The advantage of concrete execution is that DART can fall back on concrete semantics when symbolic ones are not supported (e.g., multiplications) DART detects standard errors such as crashes, assertion violations, and hangs.  Dart extracts the static interface of a program, consisting of global variables and inputs to the program entry function.  It concretizes calls to library functions.

%% Initializing values in DART: either randomly, or NULL or malloced values (for pointer types).  Chef and Cloud9 leaves it to the symbolic test the job of creating inputs. The basic primitives are marking allocated buffers (or integers or strings) as symbolic.
%% Dealing with environment inconsistencies in DART: Not a problem, as programs are re-executed (so no cross-talking in paths).  But assume no side effects in the program.
%% Solver in DART: lp\_solve (linear constraints of integers and reals).
%% DART was applied on C benchmarks and an implementation of the SIP protocol, finding hundreds of crashes.
%% DART uses DFS, BFS, or random.

%% \paragraph{CUTE}
%% Unit testing using concolic execution, with memory graphs as inputs.  Separates memory constraints from scalar constraints and keeps pointer constraints simple to keep the decision procedure tractable and efficient.
%% CUTE uses DFS.
%% CUTE uses its own in-house solver, built on top of lp\_solve, which adds optimizations such as incrementality.
%% CUTE supports generation of structured data by either calling sequences of operations, or solving data structure invariants (repOk operations), similar to Korat~\cite{boyapati:korat}.
%% Used for testing data structures.

\paragraph{Concrete + Symbolic Execution}

The major challenges of symbolically executing real-world targets are supporting the program interactions with its environment and the language constructs in its implementation.
%
A first solution introduced by early engines, such as DART~\cite{dart} and CUTE~\cite{cute}, is to instrument the source code of the program, using tools such as CIL~\cite{cil}, with additional statements that maintain the symbolic state as the program executes.  This approach resulted in the \emph{concolic} (\emph{conc}rete + symb\emph{olic}) execution model, where the program executes from a concrete input, while collecting symbolic constraints for each branch taken.  By negating the condition at each branching point, the engine obtains new concrete inputs used in subsequent program executions.
%
When the symbolic semantics are not avaiable, such as inside external library calls, the program execution can still proceed concrete-only.
%
By using this approach, DART and CUTE found crashes in C programs, such as protocol implementations and data structure libraries.

%% \paragraph{EXE}
%% Dedicated constraint solver STP, codesigned with EXE, optimized for reasoning about bitvectors and arrays, which accurately encode machine operations at good performance.
%% Has a flat view of memory.
%% Works by translating the C code to an instrumented format that includes checks for concrete/symbolic values, forks (UNIX forks), and checks for properties (memory errors, division by zero).
%% Introduced constraint optimization: caching, constraint independence.
%% DFS/BFS as search heuristics.
%% Found bugs in udhcpd, packet filters (BPF), pcre regexp library.

\paragraph{Specialized Solver Support}

In addition, real-world software generates constraints that are hard to reason about, containing expressions such as memory accesses from symbolic addresses, or complex arithmetic operations such as bit-wise multiplication and division.  Without solver support, a symbolic execution engine ends up concretizing the state or abandoning the path, hence losing completeness.

To address this problem, modern symbolic execution engines employ high-performance specialized constraint solving utilities, such as Z3~\cite{Z3}, STP~\cite{stp}, or CVC~\cite{cvc}.  These solvers can reason efficiently about a large set of operations commonly encountered in program execution, such as bit-wise arithmetic and array manipulation.
%
The EXE~\cite{exe} symbolic execution engine was among the first to employ such a solver (STP), which was co-designed with EXE to accurately express machine operations in low-level languages such as C.
%
For example, EXE modeled the program memory as a flat array of bytes, with support for arbitrary pointer reads and writes, mixed with fixed-width integer operations.
%
As a result, EXE found bugs in system software such as the \codebit{udhcpd} DHPC server, packet filters, and the \codebit{pcre} Perl regular expression library.

\paragraph{Low-level Symbolic Interpretation}

Implementing symbolic execution semantics through source code instrumentation becomes difficult for large software with multiple units, or for more complex languages, such as C++.
%
Instead, the current state-of-the-art symbolic execution engines target lower-level compiled representations of programs, such as x86 binary or LLVM~\cite{llvm} bytecode~\cite{godefroid:fuzz,klee,bitBlaze,s2eSystem,mayhem}, which consist of a smaller set of simpler instructions, with an extensive and stable specification.

%% \paragraph{SAGE}~\cite{sage2012,godefroid:fuzz}
%% Used in production, during Windows development, found thousands of vulnerabilities.  Largest known deployment of symbolic execution in production.
%% Targeted large applications, binary-level.
%% Introduces the generational search heuristic for finding bugs more efficiently in an incomplete search.

%% SAGE~\cite{sage2012,godefroid:fuzz} is the most successful use of symbolic execution in production.  It relies on concolic execution done at binary (x86) level.  It targets large applications with deep execution paths, such as Windows format parses and Microsoft Office applications.  To effectively explore the large search space, SAGE introduced a generational search heuristic.  The idea is to start from a concrete input that takes the program down a relevant path (e.g., a valid Office document), then alternate all branches on that path, then do the same with the second-generation inputs (hence the ``whitebox fuzzing'' name).
%% %
%% SAGE found thousands of Windows vulnerabilities at development time.

For example, SAGE~\cite{sage2012,godefroid:fuzz}---the most successful use of symbolic execution in production---performs concolic execution at binary level.
%
SAGE targets large applications with deep execution paths, such as Windows format parsers and Microsoft Office applications.
%
SAGE found thousands of Windows vulnerabilities at development time.

%% \paragraph{KLEE}
%% Uses symbolic execution for testing real-world systems code (Coreutils, Hi-star kernel, Busybox).  They built a high-performance symbolic execution engine that finds bugs in highly tested code and achieves high coverage levels.
%% Compact state representation using COW.
%% Uses a high-performance SMT solver (STP), and optimizes interface to the solver (constraint caching, expression simplifications).
%% Handles the environment problem through a model of files (used by the Coreutils), and external calls into the host environment.
%% Handles C code by compiling and running LLVM.
%% Uses static heuristics to maximize coverage, plus standard strategies.
%% Command line interface for marking inputs symbolic.

%% To target systems of increasing complexity, symbolic execution engines needed to address the environment problem.
%% %
%% KLEE~\cite{klee} was the first symbolic execution engine to target system software that communicates with its environment.  KLEE provides robust support for complex system software by having the software compiled to LLVM~\cite{llvm} IR, which is then interpreted symbolically.
%% %
%% KLEE~\cite{klee} introduced environment models as approximations of the real OS interface: the models replace parts of the standard C library and provided support for files.
%% %
%% This resulted in KLEE uncovering crashes and generating test suites with over 90\% coverage on average in the popular Coreutils utilities (e.g., \codebit{ls}, \codebit{echo}), testing Busybox and an OS kernel.
%% %
%% Our Cloud9 system builds on top of KLEE, providing symbolic primtives for a large fraction of the POSIX interface, as well as adding support for parallel symbolic execution.

KLEE~\cite{klee} is arguably the most popular symbolic execution engine today, having served as a foundation for a wide range of other tools.
%
KLEE works as a symbolic interpreter for LLVM IR bytecode, obtained by compiling source programs using an LLVM-based compiler such as Clang.
%
KLEE tackles the environment problem by introducing \emph{models} that approximate the real OS interface.  The KLEE models replace parts of the standard C library and provide basic support for file operations.  Unmodeled system calls are passed on to the host environment, executed concretely on behalf of the KLEE process.
%
KLEE found bugs and generated test suites with over 90\% statement coverage on average in the popular Coreutils suite.  It also found bugs in other systems software, such as Busybox and the HiStar kernel.

S2E~\cite{s2eSystem} expanded the reach of symbolic execution by providing a full virtual machine x86 symbolic execution environment, which includes programs, libraries, the operating system, and the hardware.
%
This allows S2E to accurately run ``in-vivo'' any part of the software stack, without the need of the source code.
%
Most notably, S2E discovered vulnerabilities in several Windows device drivers, some of whom were Microsoft-certified~\cite{ddt}.

%% Modeling the environment is not always feasible.  In some cases, a component is so deeply integrated in the system that there is no clear separation between the component and the rest of the system.  Device drivers running in OS kernels are one notable example.
%% %
%% For those cases, an alternative approach to the environment problem is to bundle the program, together with all its dependencies, in one large unit executed symbolically.

%% \paragraph{AEG}~\cite{aeg}
%% Automated testing with the specific purpose of generating proofs of vulnerabilities (exploits), i.e., inputs that explicitly hijack the program control flow and get a shell.
%% Takes as input LLVM + binary, produces exploit.
%% Generated 16 exploits on 14 open-source projects.
%% Buggy path-first heuristic: When a path has a bug (unexploitable), it is likely that it'll exhibit more bugs that could be exploitable.
%% Loop exhaustion strategy: prioritize paths exploring maximum number of iterations.
%% Symbolic files: simplifies KLEE's interface.
%% Symbolic sockets: supplies fresh symbolic data.
%% Environment variables (concrete values, fully symbolic, failures).
%% Network, multithreading, formatting functions (about 70 system calls).  Not clear how sound the support is, nor what happens when a syscall is not supported.

%% \paragraph{Mayhem}~\cite{mayhem}
%% Finds 29 exploitable vulnerabilities in Linux and Windows programs.
%% System designed to handle efficiently (at the expense of completeness) large state spaces generated by large programs.  Efficient reasoning about symbolic memory (symbolic writes are concretized, symbolic reads are replaced with nested ite expressions).
%% Uses only binaries (no debug info needed).

%% \paragraph{Bitblaze}~\cite{bitBlaze}
%% Unified binary analysis platform, used mainly for malware analysis.


Symbolic execution also found use in security testing~\cite{aeg,mayhem,bitBlaze}.  For this task, reaching vulnerabilities is more important than achieving completeness, so engines in this space resort to simplifications to increase performance and reach deeper executions in larger programs, at the expense of losing completeness.
%
For example, the AEG~\cite{aeg} tool uses symbolic analysis to both find bugs and automatically generate exploits (get a shell) from the bugs.  AEG uses heuristics that prioritize buggt paths and employs simple operating system models that minimize path explosion.
%
The Mayhem~\cite{mayhem} tool employs a simplified symbolic memory model where write addresses are concretized.
%
Both tools found exploitable vulnerabilities in Linux and Windows programs.

%% Handling real-world targets: C, C++ (KLOVER), parallel systems (explain the two ways to run these systems symbolically, depending on what bugs we're looking for).

\subsection{Virtual Machine-based Languages: Java and C\#}

%% \paragraph{Java Path Finder}
%% \cite{jpf-symbex,jpf-testgen,generalized-symbex}.
%% Code that manipulates complex data structures.  Uses lazy initialization to instantiate data structures.
%% Built on top of the JVM.
%% Uses iterative deepening combined with DFS.
%% Model checking as a form of testing: since the program environment is typically way too large, model checking ends up being testing.
%% Can be used to either execute a repOk method and generate blackbox inputs, or do whitebox testing.

%% \paragraph{Pex}
%% Pex~\cite{tillmann-pex}.
%% Whitebox fuzzing for .NET.  Integrated in Visual Studio.  Found errors in a core .NET component.
%% Uses PUTs (parameterized unit tests)~\cite{tillmann-puts}.
%% Uses a ``meta-strategy'' that groups branches in equivalence classes, and then selects the lest chosen class.  Different sets of equivalence classes are chosen uniformly.
%% Constructing objects: run symbolically the object constructor.

Beyond low-level system software, a significant amount of software is written in higher-level languages, offering features such as garbage collection, reflection, and built-in data structures.

Some of these languages, such as Java and C\#, are compiled to a lower-level bytecode representation and executed in a virtual machine.
%
The bytecode format is standardized and well-documented, which facilitates the development of dedicated symbolic execution engines for their runtimes.
%
For example, the Java PathFinder project~\cite{visser-jpf,jpf-symbex,jpf-testgen} provides a model checker and symbolic execution framework for Java bytecode.
%
Similarly, Pex~\cite{tillmann-pex} is a symbolic execution engine for .NET that has recently been distributed as part of Visual Studio.

The high-level environments pose additional challenges for symbolic execution related to the complexity of their data and execution model.  For example, symbolic program inputs can be both scalar and object-based.
%
Java PathFinder handles symbolic object inputs using a technique called generalized symbolic execution~\cite{generalized-symbex}, where symbolic objects are lazily initialized in response to the member accessess encountered during program execution.  Pex takes a similar approach for handling symbolic inputs in the .NET runtime.

\subsection{Dynamic Interpreted Languages}

The high-level languages that are never compiled, but source-interpreted, such as Python, JavaScript, Ruby, or Lua, are significantly more challenging for symbolic execution.
%
These languages have complex semantics that are under-specified, their features evolve rapidly from one version to another, and they rely on large built-in support libraries~\cite{dom2011,cutie-py,pythonReference}.
%
Building and maintaining a symbolic execution engine for these languages is a significant engineering effort.  As a result, the existing engines target only domain-specific subsets of their languages.

\paragraph{Supporting Symbolic Semantics}

Existing work addresses the problem of providing language semantics for symbolic execution in three major ways: by writing a symbolic interpreter for the language statements~\cite{nice}, executing the program concolically~\cite{cutie-py,jalangi}, and by requiring program cooperation~\cite{commuter}.

Writing a symbolic interpreter from scratch implies providing complete semantics for all the language constructs used by the target programs.  For example, the NICE-PySE~\cite{nice} symbolic execution engine, which is part of the NICE framework for testing OpenFlow applications, interprets the internal Python bytecode instructions generated by the interpreter when executing a program.  NICE-PySE is a Python application itself and uses the language reflection mechanisms to obtain, instrument and interpret the internal representation of the program.

The downside of writing a complete interpreter from scratch---including a symbolic execution engine---is that whenever a construct is not supported, the interpreter would halt.  To mitigate this, other engines take the concolic execution approach, where they run the real interpreter in concrete mode, and maintain in parallel the symbolic semantics of the language statements.  On the path segments where the symbolic semantics are not available, the execution can still make progress concretely, keeping the program state sound.  CutiePy~\cite{cutie-py} uses the tracing API of the Python interpreter to maintain the symbolic state in lockstep with the concrete execution.  The Jalangi dynamic instrumentation framework for JavaScript~\cite{jalangi} rewrites the target JavaScript program to insert statements for maintaining the symbolic state, similar to other symbolic execution engines for C programs~\cite{dart,cute,exe}.

There exist applications of symbolic execution where the high-level language is used only as a DSL for defining a model.  In that case, the engine only needs to support the operations explicitly defined by the engine API.  For example, the symbolic execution engine of the scalability testing tool Commuter~\cite{commuter} is entirely built in Python and offers an API for symbolically modeling operating system calls using the Python language.

In contrast to the existing techniques, Chef is the first system to use symbolic execution on an interpreter to symbolically execute a program written in the target interpreted language.  The closest work related to our approach is PokeEMU~\cite{hifi-lofi}, which uses symbolic execution on a reference CPU emulator to generate a test suite for checking the correctness of another emulator.  The generated test suite captures the semantics of the CPU instruction set, as implemented by the reference emulator.  However, the goal of PokeEMU is to check emulator implementations, while in the case of Chef, we target the programs running in the interpreter.

\paragraph{Effectiveness of Symbolic Execution for Interpreted Languages}

Symbolic execution is most known for finding lower-level bugs, such as memory errors or assertion violations.  However, the existing engines for interpreted languages show the potential of technique for finding bugs in areas where these languages are increasingly popular, such as web application security~\cite{saxena-kudzu,artzi-apollo, kiezun-ardilla}.

For instance, the Kudzu~\cite{saxena-kudzu} symbolic execution engine for JavaScript was used to detect code injection vulnerabilities.  The Apollo~\cite{artzi-apollo} engine targets PHP code to detect runtime and HTML errors, while Ardilla~\cite{kiezun-ardilla} discovered SQL injection and cross-site scripting vulnerabilities in PHP applications.
%
The hangs and unexpected exceptions discovered by the Chef-generated engines for Python and Lua also confirm the potential of symbolic execution for bug finding (Section~\ref{sec:xxx}).


%%%%%%%%%%%%%%%%%%%%%%%%%%%%%%%%%%%%%%%%%%%%%%%%%%%%%%%%%%%%%%%%%%%%%%%%%%%%%%%%

\section{Approaches to the Environment Problem}
\label{sec:relwork:envproblem}

In this section, we focus on how existing symbolic execution engines approach the environment problem, and how our contributions build upon it.

In order to efficiently achieve the testing goals for the target program, symbolic execution engines need to minimize the time spent in the environment, while ensuring correct behavior in the program.
%
The existing approaches fall into three categories: concretizing the calls into the environment, abstracting away the environment complexity through models, and ``inlining'' the environment with the program.

\subsection{Over-approximation (Concretization)}

%% Concretizing calls into the environment can be easily done in concolic mode, by simply dropping the symbolic operations while the execution carries outside the program [cite CUTE, DART, SAGE].  In pure symbolic mode, an alternative is to call the environment of the host [cite KLEE].  The disadvantage is that different execution paths can interfere for stateful calls (e.g., resource operations).

A simple approach to simplifying the environment execution is to concretize the values that cross the program boundary~\cite{dart,godefroid:fuzz,klee}.
%
For instance, when a symbolic value $\mu$ is passed as a parameter to an external function, the symbolic execution engine replaces the symbolic expression with a concrete example $m$.  This causes the execution to proceed linearly inside the environment.

However, concretization may cause inconsistencies in the execution and missing feasible paths.
%
First, the value returned from the environment is concrete, which may lead the symbolic execution engine to miss value-dependant paths in the program, such as error handling.

Second, in order to maintain soundness, the concretization introduces in the path condition of the execution state the fact that the symbolic value was constrained to one of its concrete examples ($pc_{new} := pc_{old} \wedge (\mu = m)$).  This also restricts the possible executions of the paths following the external call.  Removing the step might increase the coverage of the target program, at the expense of introducing false positives.

Finally, concretization may introduce additional inconsistencies when the environment state is shared among all execution paths, as it is the case with non-concolic (``pure-symbolic'') execution.  In that case, calls into the external environment end up operating on the same environment, causing cross-talking between different execution paths~\cite{klee}.

\subsection{Abstraction (Modeling)}

%% The last approach is to replace a concrete interface implementation with a simplified one, as either part of the symbolic execution engine (e.g., symbolic hardware in S2E), or as guest code running at the same level as the target program (e.g., uClibc models in KLEE).  The simpler the model, the less path explosion, but may introduce unsoundness or missed bugs.

An alternative approach to simplifying the environment is to replace its concrete implementation with a simpler one---a model---that captures its essential behavior.
%
This often results in the model being significantly smaller than the original, as it drops non-functional requirements, such as performance or fault-tolerance.  For example, when testing a web application, its key-value store database could be replaced with a model consisting of a plain local dictionary data structure.

Symbolic execution engines implement models either by building them into the engine implementation (e.g., symbolic hardware models in S2E~\cite{s2eSystem}), or as guest code running together with the target program (e.g., the file system models in KLEE~\cite{klee}).

Although they are effective at reducing path explosion, models are expensive to write correctly and completely.  As a result, they tend to be employed only for environment interfaces with simple semantics, or when the model accuracy is less important (e.g., security testing~\cite{aeg}).

Cloud9 takes the modeling approach to the environment problem, when symbolically executing programs interacting with the operating system.  Cloud9 is the first to provide an accurate and efficient model for an operating system interface as complex as POSIX.  Compared to previous approaches, Cloud9 divides the model code into built-in core primitives and guest code, which helps keep the implementation simple, while covering much of the most common operating system abstractions.

\subsection{Inlining (Whole-system Analysis)}

%% The opposite approach is to treat the program + environment as one large target.  Full-VM symbolic execution engines take this approach.  The problem is that path explosion in the entire system reduces the focus on the target program.  S2E introduced consistency models as a principled path pruning outside the module of interest.

Finally, the approach that preserves full correctness and completeness is to bundle the environment with the program and execute it symbolically as one target.
%
Full-VM symbolic execution engines provide this approach~\cite{s2e,bitBlaze}.
%
However, this approach reduces the effectiveness of symbolic execution on the original target program, due to the path explosion in the entire system, which may be orders of magnitude larger than the program.
%
To mitigate the path explosion, S2E introduces execution consistency models, which are principles of converting between symbolic and concrete values when crossing the program boundaries, in order to control the path explosion in the environment.

Chef takes the inlining approach to the environment problem, when symbolically executing programs written in interpreted languages.  Chef is the first system to use a language interpreter as a semantics provider for symbolic execution.


%%%%%%%%%%%%%%%%%%%%%%%%%%%%%%%%%%%%%%%%%%%%%%%%%%%%%%%%%%%%%%%%%%%%%%%%%%%%%%%%

\iffalse
\section{Path Explosion Mitigation}
\label{sec:relwork:pathexpl}

Dealing with the path explosion problem: grammar-based whitebox fuzzing (abstracting inputs), MultiSE (merging states), using symbolic execution friendly primitives (OVerify), underconstrained execution (dealing with a single module at a time).

Group heuristics by their goals and their inputs (the state information they use).

Search Heuristics in Symbolic Execution

Generational search in SAGE.

Eliminate or deprioritize the paths are are less likely to lead to the exploration goal.  Goes hand in hand with prioritization heuristics.

Replace two states meeting in the CFG with one that combines the memory locations and path conditions of the two.  Doesn't lead to improvements all the time, due to increased solver overhead [cite Trevor Hansen's study].

Compositionality means executing symbolically individual functions and storing the disjunction of path conditions as a function summary.  Can be computed incrementally, in response to program executions [cite demand-driven].

Addressing the solver bottleneck: query caches, memory models (Mayhem only models symbolic reads). Guarded sets in multiSE.
\fi

\section{Parallelizing Symbolic Execution}
\label{sec:relwork:parsymbex}

To our knowledge, we are the first to scalably parallelize symbolic execution to shared-nothing clusters.
%
\cite{parallelSymbex} described an extension to Java Pathfinder (JPF) that parallelizes symbolic execution by using parallel random searches on a static partition of the execution tree.  JPF pre-computes a set of disjoint constraints that, when used as preconditions on a worker's exploration of the execution tree, steer each worker to explore a subset of paths disjoint from all other workers.  In this approach, using constraints as preconditions imposes, at {\em every} branch in the program, a solving overhead relative to exploration without preconditions.  The complexity of these preconditions increases with the number of workers, as the preconditions need to be more selective.  Thus, per-worker solving overhead increases as more workers are added to the cluster.  This limits scalability: the largest evaluated program had 447 lines of code and did not interact with its environment.  Due to the {\em static} partitioning of the execution tree, total running time is determined by the worker with the largest subtree.  As a result, increasing the number of workers can even increase total test time instead of reducing it~\cite{parallelSymbex}.  \cnine mitigates these drawbacks.

There has been work on parallel model checking~\cite{parallelMurphi,distributed-spin,loadBalModelchecking,spin:multicore-modelchecking,modelCheckBDD}.  The SPIN model checker has been parallelized two-way for dual-core machines~\cite{parallelSPIN}. Nevertheless, there are currently no model checkers that can scale to many loosely connected computers, mainly due to the overhead of coordinating the search across multiple machines and transferring explicit states. \cnine uses an encoding of states that is compact and enables better scaling.

%%%%%%%%%%%%%%%%%%%%%%%%%%%%%%%%%%%%%%%%%%%%%%%%%%%%%%%%%%%%%%%%%%%%%%%%%%%%%%%%

\section{Defining Tests in Symbolic Execution}
\label{sec:relwork:symtests}

The user interface and API of a symbolic execution engine play an important role in limiting path explosion, as they let users define the inputs and sources of nondeterminism that can be symbolic.  This is particularly relevant for system software, whose execution is influenced by many external factors, such as command line arguments, environment variables, system call results, thread schedules, signal dispatches, and so on.

The KLEE~\cite{klee} symbolic execution engine provides an interface for controlling the symbolic environment of the target system.  For example, command-line utilities can be run with symbolic arguments that are defined using a special syntax, in line with the rest of the target arguments.  For instance, \codebit{ls -l --sym-arg 1 3} runs \codebit{ls} with the second argument symbolic, between 1 and 3 characters.  This syntax can be used to define symbolic arguments or symbolic files.

The symbolic test interface of Cloud9 gives extensive control over the behavior of its operating system model, such as per-file descriptor symbolic fault injection or per-socket symbolic control flow.
%
The closest concept to symbolic tests are the idea of parameterized unit tests~\cite{tillmann-puts}. These extend regular unit tests with parameters marked as symbolic inputs during symbolic execution.
%
QuickCheck~\cite{quickcheck} allows writing specifications in Haskell, which again share their basic concept with symbolic tests, and tries to falsify them using random testing.  Symbolic execution can offer an alternative to random testing in evaluating QuickCheck test specifications.

%%%%%%%%%%%%%%%%%%%%%%%%%%%%%%%%%%%%%%%%%%%%%%%%%%%%%%%%%%%%%%%%%%%%%%%%%%%%%%%%

\iffalse
\section{UNSORTED}

Closely related to symbolic execution is the concept of bounded model checking (BMC).  Instead of exploring individual execution paths and generate test cases, a BMC unfolds the control flow graph of the program and constructs a verification condition---a formula encompassing the behavior of the entire program with respect to a property to be checked.  Popular model checkers include CBMC~\cite{cbmc}, LLBMC~\cite{llbmc2012}, F-Soft~\cite{f-soft}, Magic~\cite{magic}, or Saturn~\cite{saturn}.

Beyond symbolic execution, there is substantial work done in the field of model checking and formal methods in general, which goes beyond the scope of this thesis.  We refer the interested reader to survey papers that cover the topic in more breadth~\cite{jhala2009software, woodcock2009formal}.

Black-box (random) fuzzing.

Unsound approaches.

Cooperative symbolic execution (where the program uses special APIs of the engine).

OVerify.
\fi

%% \paragraph{Others}

%% Other older test input generation tools: \cite{genptrinputs}.

%% Korat~\cite{boyapati:korat}. Symstra~\cite{xie:symstra}.


%%% Local Variables: 
%%% mode: latex
%%% eval: (visual-line-mode)
%%% fill-column: 1000000
%%% TeX-master: "main"
%%% End:
