\chapter*{Abstract (German)}

Manuelles Testen von Software ist aufwändig und fehleranfällig. Dennoch ist es die unter Fachleuten beliebteste Methode zur Qualitätssicherung. Die Automatisierung des Testprozesses verspricht eine höhere Effektivität insbesondere zum Auffinden von Fehlern in Randfällen. \emph{Symbolische Softwareausführung} zeichnet sich als automatische Testtechnik dadurch aus, dass sie keine falsch positiven Resultate hat, mögliche Programmausführungen abschliessend aufzählt, und besonders interessante Ausführungen prioritisieren kann. In der Praxis erschwert jedoch die sogenannte Path Explosion – die Tatsache, dass die Anzahl Programmausführungen im Verhältnis zur Programmgrösse exponentiell ansteigt – die Anwendung von Symbolischer Ausführung, denn Software besteht heutzutage oft aus Millionen von Zeilen Programmcode.

Um Software effizient symbolisch zu testen, nutzen Entwickler die Modularität der Software  und testen die einzelnen Systemkomponenten separat. Eine Komponente benötigt jedoch eine Umgebung, in der sie ihre Aufgabe erfüllen kann. Die Schnittstelle zu dieser Umgebung muss von der symbolischen Ausführungsplattform bereitgestellt werden, und zwar möglichst effizient, präzis und komplett. Dies ist das \emph{Umgebungsproblem}. Es ist schwierig, das Umgebungsproblem ein für alle mal zu lösen, denn seine Natur hängt von der gegebenen Schnittstelle und ihrer Implementierung ab.

Diese Doktorarbeit behandelt zwei Fälle des Umgebungsproblems in symbolischer Ausführung, welche zwei Extreme im Spektrum der \emph{Schnittstellenstabilität} abdecken. (1) Systemprogramme, welche mit einem Betriebssystem interagieren, dessen Semantik stabil und gut dokumentiert ist (z.B. POSIX); (2) Programme, welche in höheren, dynamischen Sprachen wie Python, Ruby oder JavaScript geschrieben sind, deren Semantik und Schnittstellen sich laufend weiterentwickeln.

Der Beitrag dieser Doktorarbeit zum Lösen des Umweltproblems im Fall von stabilen Betriebssystemschnittstellen ist die Idee, das Modell des Betriebssystems in zwei Teile zu trennen: Ein Kern von primitiven Funktionen, welche direkt in die Hostumgebung integriert sind, und darauf aufbauend eine vollständige Emulation des Betriebssystems innerhalb der zu testenden Gastumgebung. Bereits zwei primitive Funktionen genügen, um eine komplexe Schnittstelle wie POSIX zu unterstützen: Threads mit Synchronisation, und Adressräume mit gemeinsam genutzten Speicher. Unser Prototyp dieser Idee ist die symbolische Ausführungsplattform \emph{\cnine}. Ihr genaues und effizientes Modell der POSIX-Schnittstelle deckt Fehler in Systemprogrammen, Webservern und verteilten Systemen auf, welche anderweitig schwer zu reproduzieren sind. \cnine ist unter {\urlstyle{same}\url{http://cloud9.epfl.ch}} verfügbar.

Für Programme in dynamischen Sprachen stellt diese Arbeit die Idee vor, den Interpreter der Programmiersprache als „ausführbare Spezifikation” zu verwenden. Der Interpreter läuft in einer maschinennahen symbolischen Testumgebung (z.B. auf der Ebene von X86) und führt das zu testende Programm aus. Das Gesamtsystem wird dadurch zu einer symbolischen Testumgebung auf der Ebene der dynamischen Sprache. Um die Komplexität zu bewältigen, die durch das Ausführen des Interpreters entsteht, wird in dieser Arbeit die klassenuniforme Pfadanalyse (CUPA, class-uniform path analysis) eingeführt, eine Heuristik zur Prioritisierung von Pfaden. CUPA gruppiert Pfade in Äquivalenzklassen basierend auf Zielvorgaben der Analyse. Wir verwirklichten diese Ideen in unserem Prototyp \emph{\chef}, einer symbolischen Ausführungsumgebung für interpretierte Sprachen. \chef generiert bis zu 1000 Mal mehr Testfälle für populäre Python- und Luapakete als die naïve Ausführung des Interpreters. \chef ist unter  {\urlstyle{same}\url{http://dslab.epfl.ch/proj/chef}} verfügbar.

\noindent \textbf{Keywords:} Symbolische Ausführung, Programmumgebungen, Systemsoftware, interpretierte Programmiersprachen.

%%%%%%%%%%%%%%%%%%%%%%%%%%%%%%%%%%%%%%%%%%%%%%%%%%%%%%%%%%%%%%%%%%%%%%%%%%%%%%%%

\chapter*{Abstract}

Manual software testing is laborious and prone to human error.  Yet, among practitioners, it is the most popular method for quality assurance.  Automating the test case generation promises better effectiveness, especially for exposing corner-case bugs.
%
\emph{Symbolic execution} stands out as an automated testing technique that has no false positives, it eventually enumerates all feasible program executions, and can prioritize executions of interest.
%
However, path explosion---the fact that the number of program executions is typically at least exponential in the size of the program---hinders the applicability of symbolic execution in the real world, where software commonly reaches millions of lines of code.

In practice, large systems can be efficiently executed symbolically by exploiting their modularity and thus symbolically execute the different parts of the system separately.
%
However, a component typically depends on its environment to perform its task.  Thus, a symbolic execution engine needs to provide an environment interface that is efficient, while maintaining accuracy and completeness.  This conundrum is known as \emph{the environment problem}.
%
Systematically addressing the environment problem is challenging, as its instantiation depends on the nature of the environment and its interface.

This thesis addresses two instances of the environment problem in symbolic execution, which are at opposite ends of the spectrum of \emph{interface stability}: (1) system software interacting with an operating system with stable and well-documented semantics (e.g., POSIX), and (2) high-level programs written in dynamic languages, such as Python, Ruby, or JavaScript, whose semantics and interfaces are continuously evolving.

To address the environment problem for stable operating system interfaces, this thesis introduces the idea of splitting an operating system model into a core set of primitives built into the engine at host level and, on top of it, the full operating system interface emulated inside the guest.
%
As few as two primitives are sufficient to support a complex interface such as POSIX: threads with synchronization and address spaces with shared memory.
%
We prototyped this idea in the \emph{\cnine} symbolic execution platform. \cnine's accurate and efficient POSIX model exposes hard-to-reproduce bugs in systems such as UNIX utilities, web servers, and distributed systems.
%
\cnine is available at {\urlstyle{same}\url{http://cloud9.epfl.ch}}.

For programs written in high-level interpreted languages, this thesis introduces the idea of using the language interpreter as an ``executable language specification''.  The interpreter runs inside a low-level (e.g., x86) symbolic execution engine while it executes the target program.  The aggregate system acts as a high-level symbolic execution engine for the program.
%
To manage the complexity of symbolically executing the entire interpreter, this thesis introduces Class-Uniform Path Analysis (CUPA), an algorithm for prioritizing paths that groups paths into equivalence classes according to a coverage goal.
%
We built a prototype of these ideas in the form of \emph{\chef}, a symbolic execution platform for interpreted languages that generates up to 1000 times more tests in popular Python and Lua packages compared to a plain execution of the interpreters.
%
\chef is available at {\urlstyle{same}\url{http://dslab.epfl.ch/proj/chef/}}.

\noindent \textbf{Keywords:} Symbolic execution, program environments, systems software, interpreted languages.


%%% Local Variables: 
%%% mode: latex
%%% eval: (visual-line-mode)
%%% fill-column: 1000000
%%% TeX-master: "main"
%%% End:
