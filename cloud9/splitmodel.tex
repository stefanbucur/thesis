In this section, we first present the design space of an operating system model (Section~\ref{sec:cloud9:modelspace}).  We then present our approach of a split model (Section~\ref{sec:cloud9:approach}) and how we leverage it using symbolic tests.


\subsection{The Design Space of Operating System Models}
\label{sec:cloud9:modelspace}

The goal of a symbolic model is to simulate the behavior of a real execution environment, while maintaining the necessary symbolic state behind the environment interface.
%
The symbolic execution engine can then seamlessly transition back and forth between the program and the environment.

Writing and maintaining a model can be laborious and prone to error~\cite{s2eSystem}.
%
However, for operating system interfaces, which are typically stable and well documented, investing the time to model them becomes worthwhile.

Ideally, a symbolic model would be implemented as code that gets executed by the symbolic execution engine (i.e., guest-level code), substituting the implementation of the operating system and the C library.
%
Such a model can be substantially faster.  For instance, in the Linux kernel, transferring a packet between two hosts exercises the entire TCP/IP networking stack and the associated driver code, amounting to over 30 \kloc.  In contrast, our POSIX model achieves the same functionality in about 1.5 \kloc.  Requirements that complicate a real environment/OS implementation, such as performance and extensibility, can be ignored in a symbolic model.

However, not all operating system abstractions can be directly expressed as guest code.
%
In general, the problematic abstractions are those \emph{incompatible} with the execution model of the symbolic execution engine.
%
For example, providing support for multiple execution threads may not be achievable in the guest of a symbolic execution engine designed to run sequentially, unless the guest can manipulate the current stack and program counter.
%
There are other abstractions incompatible with the typical sequential single-process execution model of a symbolic execution engine, such as processes, synchronization, and shared memory.

A possible alternative is to implement the operating system model inside the symbolic execution engine, where we can define any abstraction.
%
However, this approach is undesirable.  A model built into a symbolic execution engine is significantly more expensive to produce, because the model code has to handle explicitly symbolic data and the program state.
%
For example, while a guest model of the \codebit{open} system call could directly use the string buffer of the file name passed as argument, a built-in model needs to explicitly invoke read operations for the string bytes, then extract the character values from the symbolic AST expressions.


\subsection{A Split Operating System Model}
\label{sec:cloud9:approach}

Our key idea is to take the best from both worlds and provide an operating system model that is \emph{split} between a set of built-in primitives and guest-level code.
%
The built-in primitives model only the minimal operating system abstractions that would be more expensive or impossible to model at the guest level.
%
In turn, the guest-level code implements a complete operating system interface on top of these primitives.
%
In analogy to the system calls exposed by an operating system to user programs, the symbolic execution engine provides its primitives to the guest through a set of ``symbolic system calls''.

The built-in primitives provide the operating system abstractions that depend on the execution model of the symbolic execution engine.
%
The execution model includes aspects such as the program memory model and the control flow mechanisms.
%
For performance reasons, these aspects are typically encoded in the logic of the symbolic execution engine.

We identified two abstractions that are prevalent in operating systems and that should be built into the symbolic execution engine: \emph{multithreading} and \emph{address spaces}.
%
Multithreading is best provided as a built-in primitive that is integrated with the control flow mechanisms of the symbolic execution engine.
%
Similarly, providing multiple isolated address spaces is best provided by the memory model of the symbolic execution engine.  For example, the functionality needed to support address spaces (e.g., cloning) is shared with the requirements of cloning the execution state after a symbolic branch.

Many other common abstractions do not need to be built into the symbolic execution engine, but can be emulated as guest code on top of threads and address spaces.
%
Such derived abstractions include mechanisms for inter-process communication, such as sockets, pipes and files.
%
Various synchronization mechanisms, such as mutexes, semaphores, condition variables, and barriers can be provided on top of basic multi-threading primitives that put a thread to sleep and wake it up later.

We designed the support for multithreading and address spaces according to two goals: (1) minimizing the complexity of the guest code using them, and (2) capturing all possible behaviors in the real operating system, including corner-case, hard-to-reproduce scenarios.
%
The latter is especially relevant when using the symbolic execution engine to find bugs occurring in the program interaction with the operating system, as we later show in the evaluation (Section~\ref{sec:eval:bug-finding}).


\subsection{Built-in Primitives: Multithreading and Address Spaces}
\label{sec:cloud9:primitives}

We next describe the design of multithreading support and address spaces in a symbolic execution engine.

\paragraph{Multithreading}

%% A symbolic execution engine is essentially an interpreter of program statements.
%% %
To provide support for multiple threads, the symbolic execution engine maintains in each execution state a set of per-thread stacks, holding the current program location, the execution call chain, and local variables.
%
During symbolic execution, the execution alternates between the threads, governed by a thread scheduler built into the symbolic execution engine.

To simplify synchronization inside the guest model, we use a cooperative scheduler.  An enabled thread runs uninterrupted (atomically), until either (a) the thread goes to sleep, (b) the thread is explicitly preempted, or (c) the thread is terminated with a symbolic system call.

The scheduler can be configured to schedule the next thread deterministically, or to fork the execution state for each possible next thread.
%
The latter case is useful when looking for concurrency bugs.  At the same time, it can be a significant source of path explosion, so it can be selectively disabled when not needed.

The symbolic execution engine can detect hangs in the system, such as deadlocks, when a thread goes to sleep and no other thread can be scheduled.

\paragraph{Address Spaces}

In symbolic execution, program memory is typically represented as a mapping from memory locations (e.g., variables or addresses) to slots holding symbolic expressions.
%
To provide support for address spaces, we built into the symbolic execution engine support for multiple such mappings per execution state.

Each thread in the execution state is bound to one address space and each address space with its threads forms a process in the execution state.

The symbolic execution engine provides a symbolic system call for \emph{sharing} slots among multiple memory mappings.  This mechanism provides the foundation for implementing shared memory across multiple processes.
%
Shared memory can be used by the guest model to provide multiple forms of inter-process communication, such as sockets, files, and pipes.
%
For example, a socket can be modeled as a pair of memory buffers, one of each direction, shared between the client and the server processes.


\paragraph{Symbolic System Calls}

Table~\ref{table:cloud9:primitives} shows the symbolic system calls that the engine provides to the guest to support multithreading and address spaces.  We detail below the most important system calls.

\begin{table}
\centering
  
\begin{tabular}{|l|l|}
\hline
\textbf{Primitive Name} & \textbf{Description} \\
\hline
\hline
 \codebit{thread\_create(\&func)} & Create new thread that runs \codebit{func} \\
 \codebit{thread\_terminate()} & Terminate current thread \\
\hline
 \codebit{thread\_preempt()} & Preempt the current thread  \\
\hline
 \codebit{create\_wqueue()} & Create a new waiting queue \\
 \codebit{thread\_sleep(wq)} & Put current thread to sleep on waiting queue \\
 \codebit{thread\_notify(wq)} & Wake threads from waiting queue \\
\hline
\hline
 \codebit{process\_fork()} & Fork the current address space and thread \\
 \codebit{make\_shared(\&buf, size)} & Share memory across address spaces \\
\hline
\hline
 \codebit{get\_context()} & Get the current context (process and thread ID) \\
\hline
\end{tabular}

\caption[Symbolic system calls for multithreading and address spaces.]{Symbolic system calls for multithreading (first block), address spaces (second block).  Third block contains introspection primitives.  The multithreading block is further divided into thread lifecycle management, explicit scheduling, and synchronization.}
\label{table:cloud9:primitives}
\end{table}

Threads are created in the currently executing process by calling \codebit{thread\_\allowbreak{}create}.  For instance, the POSIX threads (pthreads) model makes use of this primitive in its own \codebit{pthread\_create()} routine.
%
When \codebit{thread\_sleep} is called, the symbolic execution engine places the current thread on a specified waiting queue, and an enabled thread is selected for execution.
%
Another thread may call \codebit{thread\_\allowbreak{}notify} on the waiting queue and wake up one or all of the queued threads.

Cloning the current address space is available to the guest through the \codebit{process\_fork} primitive, which is used, for instance, to model the POSIX \codebit{fork()} call.
%
A memory location can be marked as shared by calling \codebit{make\_\allowbreak{}shared}; it is then automatically mapped in the address spaces of the other processes in the execution state.  Whenever a shared object is modified in one address space, the new version is automatically propagated to the others.


%%% Local Variables: 
%%% mode: latex
%%% eval: (visual-line-mode)
%%% fill-column: 1000000
%%% TeX-master: "main"
%%% End:
