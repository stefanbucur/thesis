Software systems typically have large ``hand-made'' test suites.
%
Writing and maintaining these suites requires substantial human effort, for two main reasons: (1) the developers need to devise a comprehensive set of concrete inputs to cover the program behaviors and (2) setting up the environment (i.e., the operating system) to expose all relevant program interactions is difficult.
%
We aim to reduce both burdens, while improving the quality of testing, by introducing \emph{symbolic tests}.

A symbolic test is a piece of code executed by the symbolic execution engine, which sets up the target program, creates symbolic inputs, and configures the operating system environment.
%
A symbolic test encompasses many similar concrete test cases into a single symbolic one---each symbolic test a developer writes is equivalent to many concrete ones.
%
Furthermore, symbolic tests can easily uncover corner cases, as well as new, untested functionality.

A symbolic test controls the operating system model to explore conditions that are hard to produce reliably in a concrete test case, such as the occurrence of faults, concurrency side effects, or network packet reordering, dropping, and delay.

In this section, we present the developer interface for writing symbolic tests and illustrate it with a use case.

\subsection{Testing Platform Interface}

The symbolic testing API (Tables~\ref{table:globalapi} and~\ref{table:ioctlapi}) allows tests to programmatically control events in the environment of the program under test.
%
A test suite needs to simply include a header file and make the requisite calls.

\begin{table}
\centering
\begin{tabular}{|l|l|}
\hline
\textbf{Function Name} & \textbf{Description} \\
\hline
\codebit{make\_symbolic} & Mark memory regions as symbolic \\
\hline
\codebit{fi\_enable} & Enable/disable the injection of faults \\
\codebit{fi\_disable} & \\
\hline
\codebit{set\_max\_heap} & Set heap size for symbolic \codebit{malloc} \\
\hline
\codebit{set\_scheduler} & Set scheduler policy (e.g., round-robin)\\
\hline
\end{tabular}
\caption{API for setting global behavior parameters.}
\label{table:globalapi}
\end{table}

\begin{table}
\centering
\begin{tabular}{|l|l|}
\hline
\textbf{Extended Ioctl Code} & \textbf{Description} \\
\hline
\codebit{SIO\_SYMBOLIC} & Turn a file or socket into a source of symbolic input \\
\hline
\codebit{SIO\_PKT\_FRAGMENT} & Enable packet fragmentation on a stream socket\\
\hline
\codebit{SIO\_FAULT\_INJ} & Enable fault injection for operations on descriptor \\
\hline
\end{tabular}

\caption{Extended \codebit{ioctl} codes to control environmental events on a per-file-descriptor basis.}
\label{table:ioctlapi}
\end{table}

\paragraph{Symbolic Data and Streams}

The generality of a test case can be expanded by introducing bytes of symbolic data.
%
This is done by calling \codebit{make\_symbolic} to mark data symbolic, a wrapper around the symbolic execution engine's primitive for injecting fresh symbolic variables in the program state.

In addition to wrapping this call, we added several new primitives to the testing API.  Tables~\ref{table:globalapi} and \ref{table:ioctlapi} show how these primitives are provided in a POSIX environment.
%
Symbolic data can be written/read to/from files, can be sent/received over the network, and can be passed via pipes.
%
Furthermore, the \codebit{SIO\_SYMBOLIC} \codebit{ioctl} code (Table~\ref{table:ioctlapi}) turns on/off the reception of symbolic bytes from individual files or sockets.

\paragraph{Network Conditions}

Delay, reordering, or dropping of packets causes a network data stream to be fragmented.
%
Fragmentation can be turned on or off at the socket level, for instance, by using one of the \codebit{ioctl} extensions.  Section~\ref{sec:eval:lighttpd} presents a case where symbolic fragmentation proved that a bug fix for the lighttpd web server was incomplete. 

\paragraph{Fault Injection}

Operating system calls can return an error code when they fail.
%
Most programs can tolerate such failed calls, but even high-quality production software misses some~\cite{lfi}. Such error return codes are simulated in a symbolic test whenever fault injection is turned on. 

\paragraph{Symbolic Scheduler}

The built-in multithreading primitive provides multiple scheduling policies that can be controlled for purposes of testing on a per-code-region basis.
%
Currently, it supports a round-robin scheduler and two schedulers specialized for bug finding: a variant of the iterative context bounding scheduling algorithm~\cite{chess} and an exhaustive exploration of all possible scheduling decisions.  


\subsection{Usage Example: Testing Custom Header Handling in Web Server}

Consider a scenario in which we want to test the support for a new \codebit{X-NewExtension} HTTP header, just added to a UNIX web server.
%
We show how to write tests for this new feature.

A symbolic test suite typically starts off as an augmentation of an existing test suite;
%
in our scenario, we reuse the existing boilerplate setup code and write a symbolic test case that marks the extension header symbolic. Whenever the code that processes the header data is executed, the symbolic execution engine forks at all the branches that depend on the header content. Similarly, the request payload can be marked symbolic to test the payload-processing part of the system:

\begin{verbatim}
   char hData[10];
   cloud9_make_symbolic(hData);
   strcat(req, "X-NewExtension: ");
   strcat(req, hData);
\end{verbatim}

The web server may receive HTTP requests fragmented in a number of chunks, returned by individual invocations of the \codebit{read()} system call---the web server should run correctly regardless of the fragmentation pattern.
%
To test different fragmentation patterns, one simply enables symbolic packet fragmentation on the client socket:
\begin{verbatim}
   ioctl(ssock, SIO_PKT_FRAGMENT, RD);
\end{verbatim}

To test how the web server handles failures in the environment, we configure the symbolic test to selectively inject faults when the server reads or sends data on a socket by placing in the symbolic test suite calls of the form:
\begin{verbatim}
   ioctl(ssock, SIO_FAULT_INJ, RD | WR);
\end{verbatim}
We can also enable/disable fault injection globally for all file descriptors within a certain region of the code using calls to \codebit{fi\_\allowbreak{}enable} and \codebit{fi\_\allowbreak{}disable}. For simulating low-memory conditions, we provide a \codebit{set\_\allowbreak{}max\_heap} primitive, which can be used to test the web server with different maximum heap sizes.

%%% Local Variables: 
%%% mode: latex
%%% eval: (visual-line-mode)
%%% fill-column: 1000000
%%% TeX-master: "main"
%%% End:
